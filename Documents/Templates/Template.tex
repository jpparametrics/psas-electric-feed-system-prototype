\documentclass[12pt]{article}
\usepackage[T1]{fontenc}
\usepackage{graphicx}
\usepackage[font=small]{caption}
\usepackage{amsmath,amssymb,amsfonts,amsthm}
\usepackage{eso-pic}
\usepackage{lastpage}
\usepackage[left=2.7cm,right=1.7cm,top=1.65cm,bottom=1.0cm]{geometry}

\usepackage{fontspec}

\usepackage{tkz-euclide}
\usepackage{multicol}
\usepackage{pgfplots}
\pgfplotsset{compat=1.8}
\usepackage{pgfplotstable}
\usepackage{tikz}
\usetkzobj{all}
\usepackage{booktabs}
\usepackage{listings}
\usepackage{stmaryrd}
%\usepackage{courier}
%\setmonofont{Courier New}
\lstset{
language=R,
basicstyle=\scriptsize\ttfamily,
commentstyle=\ttfamily\color{gray},
numbers=left,
numberstyle=\ttfamily\color{gray}\footnotesize,
stepnumber=1,
numbersep=5pt,
backgroundcolor=\color{white},
showspaces=false,
showstringspaces=false,
showtabs=false,
frame=single,
tabsize=2,
captionpos=b,
breaklines=true,
breakatwhitespace=false,
title=\lstname,
escapeinside={},
keywordstyle={},
morekeywords={},
morecomment=[l][\color{red}]{>\ }
}



% configure page header
\usepackage{fancyhdr}
\setlength{\headheight}{18pt}
\pagestyle{fancy}\fancyhf{}
\renewcommand{\headrulewidth}{0pt}
\renewcommand{\footrulewidth}{0pt}
\lhead{\large{11/3/16}}
\chead{\large{ME 488 - HWK \#004}\hspace{1.4cm}}
\rhead{\large{\MakeUppercase{JORDEN ROLAND}}\hspace{1.5cm}\thepage\hspace{0.1cm}/\hspace{0.1cm}\pageref{LastPage}\hspace{-1.5cm}}

% background image
\newcommand\BackgroundPic{
\put(0,0){
\parbox[b][\paperheight]{\paperwidth}{%
\vfill
\centering
\includegraphics[width=\paperwidth,height=\paperheight, keepaspectratio]{background.png}%
\vfill
}}}

% configure section titles
\usepackage{titlesec}
\titleformat{\section}{\Large\bfseries}{\thesection}{1em}{}
\titleformat{\subsection}{\large\bfseries}{\thesubsection}{1em}{}

\begin{document}

\AddToShipoutPicture{\BackgroundPic}

% custom commands
\newcommand{\windef}[1]{\subsection*{\underline{DEF:} \normalsize{#1}}} % definition
\newcommand{\winex}[1]{\subsection*{\underline{EX:} \normalsize{#1}}} % example
\newcommand{\winsec}[1]{\section*{\underline{#1}}} % section
\newcommand{\winres}[1]{\begin{math}\Rightarrow\left\{\begin{matrix}#1\end{matrix}\right.\end{math}\hspace{0.2cm}} % result
\newcommand{\winsys}[1]{\begin{math}\left\{\begin{matrix}#1\end{matrix}\right.\end{math}} % system
\newcommand{\winero}[1]{\begin{math}\xrightarrow{#1}\end{math}} % row operation
\newcommand{\winmat}[1]{\begin{math}\begin{bmatrix}#1\end{bmatrix}\end{math}} % matrix
\newcommand{\winsub}[1]{\subsection*{$\star$\hspace{0.2cm} #1}} % dot
\newcommand{\step}[1]{\begin{math}\xrightarrow{\text{#1}}\end{math}} % step in a process
\newcommand{\winrtwo}{\begin{math}\text{R}^\text{2}\end{math}}
\newcommand{\winrthree}{\begin{math}\text{R}^\text{3}\end{math}}
\newcommand{\wineq}[1]{\begin{equation}\notag#1\end{equation}}
\newcommand{\winans}[1]{\begin{equation}\underline{\underline{\notag\boldsymbol{#1}}}\Leftarrow\end{equation}}
%%%%%%%%%%%%%%%%%%%%%%
% start writing here %
%%%%%%%%%%%%%%%%%%%%%%
\winsec{Question 1: Fractional Factorials}

\winsub{Given}

\begin{enumerate}
  \item $2^{5-2}$
  \item $2^{8-4}$
  \item $2^6$
\end{enumerate}

\winsub{Find}

Runs per replicate needed for each of the above Fractional Factorials

\winsub{Solution}

\begin{enumerate}
  \item $2^{5-2} = 2^3 = 8$ runs per replicate.
  \item $2^{8-4} = 2^4 = 16$ runs per replicate.
  \item $2^6 = 64$ runs per replicate.
\end{enumerate}


\vspace{0.5in}
\noindent\hfil\rule{0.5\textwidth}{.8pt}\hfil

\winsec{Question 2: Use R to Generate Fractional Factorials}

\winsub{Given}

\begin{enumerate}
  \item Any $2^{5-1}$
  \item A $2^{8-3}$ with the generators F=ABC, G=ABD, H=BCD
\end{enumerate}

\winsub{Find}

Use R and the FrF2 Package to create the above and display the +1/-1 matrices

\newpage

\winsub{Solution}
\begin{enumerate}
\item \hspace{1.0cm}\begin{lstlisting}
> library(DoE.base)
> library(FrF2)
> FrF2(2^4,5)
    A  B  C  D  E
1  -1  1  1  1 -1
2   1  1 -1 -1  1
3  -1  1 -1  1  1
4  -1 -1  1  1  1
5   1 -1  1 -1  1
6   1  1  1  1  1
7   1 -1  1  1 -1
8  -1  1 -1 -1 -1
9   1  1 -1  1 -1
10 -1 -1 -1 -1  1
11 -1 -1  1 -1 -1
12  1  1  1 -1 -1
13  1 -1 -1 -1 -1
14 -1 -1 -1  1 -1
15 -1  1  1 -1  1
16  1 -1 -1  1  1
class=design, type= FrF2 
\end{lstlisting}
\item  \hspace{1.0cm}\begin{lstlisting}
> FrF2(2^5,8,generator=c("ABC","ABD","BCD"))
    A  B  C  D  E  F  G  H
1   1 -1 -1  1  1  1 -1  1
2  -1 -1 -1  1  1 -1  1  1
3  -1  1  1  1 -1 -1 -1  1
4   1  1  1  1  1  1  1  1
5  -1 -1  1  1  1  1  1 -1
6   1  1 -1  1  1 -1  1 -1
7   1 -1  1  1  1 -1 -1 -1
8  -1 -1  1 -1  1  1 -1  1
9  -1  1 -1  1 -1  1 -1 -1
10 -1  1  1 -1  1 -1  1 -1
11  1 -1  1 -1  1 -1  1  1
12 -1  1 -1 -1  1  1  1  1
13  1 -1 -1 -1  1  1  1 -1
14  1  1 -1  1 -1 -1  1 -1
15 -1  1 -1 -1 -1  1  1  1
16 -1 -1 -1  1 -1 -1  1  1
17 -1  1  1  1  1 -1 -1  1
18 -1 -1 -1 -1  1 -1 -1 -1
19  1 -1 -1  1 -1  1 -1  1
20  1 -1 -1 -1 -1  1  1 -1
21  1 -1  1 -1 -1 -1  1  1
22  1  1 -1 -1  1 -1 -1  1
23  1  1  1  1 -1  1  1  1
24 -1 -1  1 -1 -1  1 -1  1
25  1  1  1 -1 -1  1 -1 -1
26 -1  1  1 -1 -1 -1  1 -1
27  1 -1  1  1 -1 -1 -1 -1
28 -1 -1 -1 -1 -1 -1 -1 -1
29 -1 -1  1  1 -1  1  1 -1
30  1  1 -1 -1 -1 -1 -1  1
31 -1  1 -1  1  1  1 -1 -1
32  1  1  1 -1  1  1 -1 -1
class=design, type= FrF2.generators
\end{lstlisting}
\end{enumerate}

\newpage

\winsec{Question 3: Aliasing}

\winsub{Given}

\begin{enumerate}
  \item I=ABCD=EBCD=AE
  \item I=ABCDE
\end{enumerate}

\winsub{Find}

The aliases of 'BC' in each of the above

\winsub{Solution}

\begin{enumerate}
  \item Multiply through the defining words by `BC' gives its aliases.\\
  (BC)I=(BC)ABCD=(BC)EBCD=(BC)AE $\shortrightarrow$  BC=AD=ED=ABCE
  \item Repeat step above.\\
  (BC)I=(BC)ABCDE $\shortrightarrow$ BC=ADE
  \end{enumerate}


\vspace{0.5in}
\noindent\hfil\rule{0.5\textwidth}{.8pt}\hfil

\winsec{Question 4: Resolution}

\winsub{Find}

Explain why a $R_{II}$ fractional experiment is a bad idea? Use an example to illustrate your point.

\winsub{Solution}
In general a $R_{II}$ fractional experiment is not particularly useful because main effects will be confounded with other main effects making it unclear which of the factors is causing a change in the resonse.\\

EX: The simplest case is a $2^{2-1}$ experiment with defining relationship I=AB. In this case A=B and B=A making the two factors being tested completely indistiguishable from each other.

\newpage

\winsec{Question 5: Concepts of Half Fractional}

\winsub{Given}

$2^{4-1}$ Half Fractional Factorial Experiment with the D=ABC generator

\winsub{Find}

The process to create the 'other' half. Hint: D=ABC means D=+ABC

\winsub{Solution}
Generating the ``other half'' of the experiment simply requires setting D=-ABC as below.
\begin{lstlisting}
> FrF2(2^3,4,generator=c("-ABC"))
   A  B  C  D
1  1  1  1 -1
2 -1  1  1  1
3 -1 -1 -1  1
4  1 -1  1  1
5  1 -1 -1 -1
6  1  1 -1  1
7 -1 -1  1 -1
8 -1  1 -1 -1
class=design, type= FrF2.generators 
\end{lstlisting}



\newpage
\winsec{Question 6: Application}

\winsub{Given}

You have been given authorization to study a flame-resistant material. There are 8 key factors, (A,B,C,D,E,F,G,H)

\winsub{Find}

\begin{enumerate}
  \item How many samples at a minimum would you need to request to perform a Full Fractional Experiment of any use?
  \item You have been authorized only 100 samples at maximum. What are the feasible Balanced Fractional Experiments you could run?
  \item For each of the experiments you listed in the last part, what are the engineering trade-offs in the feasible ones?
  \item Pick your choice of experiment and state issues with that experiment you would need to keep in consideration during the analysis phase.
  \item Assume DE and BC are significant and critical two way interactions, use the FrF2 package to determine generators for your choice above to address these.
\end{enumerate}

\winsub{Solution}

\begin{enumerate}
  \item $2^8=256$ runs per replicate. Assuming at \textbf{least} 2 runs for a reasonable number of $df_{error}$ gives $256*2=512$ samples.
  
  \item \textbf{Half-fractional:}	$2^{8-1}=128$ still too many samples.\\
  \textbf{Quarter-fractional:}   $2^{8-2}=64$ This is a pretty good option. ($R_{V}$) and less than 100 samples.\\
  \textbf{Eighth-fractional:} $2^{8-3}=32$ This is probably a low number of samples but not a terrible option ($R_{IV}$). Also leaves enough room to run a second and third replicate which is never bad.\\
  \textbf{Sixteenth-fractional:} $2^{8-4}=16$ Still $R_{IV}$  so this is not a bad option really, allows for oportunity to run multiple replicates. However there may be significant 2-way interactions which are aliased.
  
  \item See above.
  \item Choosing the Eighth-fractional design gives the best options for aliasing combined with the above stated 3 replicates within the funding allowance means that this is likely the best design.
  
  \newpage
  
  
  \item Using the FrF2 function in R with the following generators: F=ABC, G=ABD, H=BCDE
  \begin{lstlisting}
  > design.info(FrF2(2^5,8,generator=c("ABC","ABD","BCDE")))$aliased
$legend
[1] "A=A" "B=B" "C=C" "D=D" "E=E" "F=F" "G=G" "H=H"
$main
character(0)
$fi2
[1] "AB=CF=DG" "AC=BF"    "AD=BG"    "AF=BC"    "AG=BD"    "CD=FG"    "CG=DF" 
  \end{lstlisting}
  As shown above DE and BC are not aliased with other interactions and mains are not confounded. this is a solid design.
  
\end{enumerate}

\noindent\hfil\rule{0.5\textwidth}{.8pt}\hfil
\begin{center}
  \textbf{END OF ASSIGNMENT}
\end{center}
\end{document}
